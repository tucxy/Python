I am putting various theorems and constructions with way too much detail here so we can trim them down.
\begin{definition}\label{sigmapm}
  \cite{FreyTran} A $\sigma^{+-}$ labeling of $G$ is an injection $f:V(G) \rightarrow \{0,1,\hdots, 2m-2\}$ which induces a bijective length function $\ell:E(G)\rightarrow \{1, 2,\hdots, m\}$ where $\ell(uv) = f(b)-f(u)$ for every edge $uv\in E(G)$ with $u\in A$ and $v\in B$, $a<b$ and $f$ has the additional property that $f(u)-f(v) \neq m$ for all $u\in A$ and $v\in B$.
\end{definition}

\begin{thm}\label{sigmadesign}
  (Freyberg and Tran, \cite{FreyTran}). Let $G$ be a bipartite graph with $m$ edges and a $\sigma^{+-}$-labeling such that the edge of length $m$ is a pendant edge $e$. Then there exists a $G$-decomposition of $K_{2mr}$ and $K_{2mr+1}$ for every positive integer $r$.
\end{thm}

\begin{observation}\label{Knlen}
  Consider $K_{n}$ for any $n\in \mathbb{N}$ and $\ell$ on it's edges as stated in Definition \ref{sigmapm}. For any edge $e\in E(K_{21}),\;\ell(e)\in \{1,\hdots,\floor{\frac{n}{2}}\}$ and there exist $n$ edges with each distinct length in  $\{1,\hdots,\floor{\frac{n}{2}}\}$. Equivalently, each vertex $v\in V(K_{21})$ is a member of exactly two distinct edges $e_{1},e_{2}$ of each length in $\{1,\hdots,\floor{\frac{n}{2}}\}$.
\end{observation}
\newpage
\begin{construction}\label{K21design}
    Let $F$ be a forest on $7$ edges and consider $K_{21}$. By \ref{Knlen}, any edge in $K_{21}$ has a length in $L=\{1,\hdots,10\}$ via $\ell$ as stated in  Definition \ref{sigmapm}. Next, by Theorem \ref{sigmadesign}, if $F_{\sigma}$ is a $\sigma^{+-}-$labeling of $F$, then $B_{\sigma}=\langle F_{\sigma}\rangle$ via $\phi_{1}:=v\mapsto v+1$ contains all edges of $K_{21}$ with lengths in $L_{\sigma}=\{1,\hdots,7\}$ by the union of it's members. Additionally, all $G\in B_{\sigma}$ are isomorphic to $F$.

    Now, clearly each edge in $K_{21}$ is uniquely determined by it's endpoints. So naturally observe that if we partition the vertices of $K_{21}$ modulo $7$, a singleton edge partition of $K_{21}$ modulo $7\times 7$ is induced. Let $\ell:=ab\mapsto a+b\Mod{7}$. We now partition the edges of $K_{21}$ modulo $\sim$ where $ab\sim cd\text{ if } \ell(ab)=\ell(cd)\text{ and }\ell_{7}(ab)=\ell_{7}(cd)$, and specify each partite set $P_{i,j}$ via $ab\in P_{i,j}$ if $\ell\times \ell_{7}(ab)=(i,j)$. Each $P_{i,j}$ contains three edges in the same equivalence class modulo $\sim$; three edges of the same length $i$ which are in the same endpoint equivalence class modulo $7\times 7$. Lastly, let $L_{i}=\{uv\in E(K_{21})\mid \ell(uv)=i\},\forall i\in L$, then  $L_{i}=\bigcup_{j\in \mathbb{Z}_{7}}P_{i,j}$.

    Let $\phi_{7}:=v\mapsto v+7$ and $H_{i,j}\subseteq K_{21}$ be a subgraph containing some edge $ab$ such that $\ell\times \ell_{7}(ab)=(i,j)$. Then if $G_{i,j}=\bigcup_{G\in \langle H_{i,j}\rangle_{\phi_{7}}}G,\;P_{i,j}\subseteq E(G_{i,j});$ each $G_{i,j}$ contains the partition $P_{i,j}$. Then since for any $(i,j)\in L\times \ZZ_{7},$ such a subgraph $H_{i,j}$ exists, so does such a $G_{i,j}$. Therefore, for any $i\in L$, $L_{i}\subseteq E(\bigcup_{j\in \ZZ_{7}}G_{i,j})$ for such $G_{i,j}$. In other words, we can generate all edges of length $i\in L$ in $K_{21}$ by finding subgraphs containing edges of length $i$ from each distinct equivalence class modulo $7\times 7$ and applying $\phi_{7}$ to their vertices.

    So finally, if there exist $\ell\times \ell_{7}-$wise edge-disjoint subgraphs $F_{1},F_{2},F_{3}\cong F$ of $K_{21}$ with only edges of lengths in $L^{*}=\{8,9,10\}$, let $B_{i}=\langle F_{i}\rangle_{\phi_{7}}$ for each $i=1,2,3$. Then $\bigcup_{i=1,2,3}B_{i}$ must contain all edges of $K_{21}$ with lengths in $L^{*}$ by the union of it's members. Let $B_{21}=B_{\sigma}\cup B_{1}\cup B_{2}\cup B_{3}$ and recall that $B_{\sigma}$ contains all edges of lengths in $L_{\sigma}$ by the union of it's members. So then since $L=L_{\sigma}\cup L^{*},\;K_{21}=\bigcup_{G\in B_{21}}G,$ and all $G\in B_{21}$ are isomorphic to $F$ and edge disjoint by definition. Therefore,
    \begin{center}
    the existence of such subgraphs $F_{\sigma},F_{1},F_{2},F_{3}$ give $B_{21}=B_{\sigma}\cup B_{1}\cup B_{2}\cup B_{3}$, an $F-$decomposition of $K_{21}.$
    \end{center}
\end{construction}

\begin{construction}\label{K22design}
    Let $F$ be a forest graph on seven edges and consider $K_{22}$. For this construction we simply take the $22$nd vertex to be $\infty$; we add $\infty$ to the neighborhood of each vertex in $K_{21}$, and let $\ell(x\infty)=\infty$ for all $x\in V(K_{22})$. Next, we let $B_{\sigma}$ refer to the same block as previously described in Construction \ref{K21design}. Similarly to in the previous construction, if $F_{1},F_{2},F_{3},F_{4}\cong F$ are $\ell\times \ell_{7}-$wise edge-wise disjoint subgraphs of $K_{21}$ with only edges of lengths in $L^{*}=\{8,9,10\}\cup \{\infty\}$, we let $B_{i}=\langle F_{i}\rangle_{\phi_{7}}$ via $\phi_{7}$ as defined in Construction \ref{K21design} with the new condition $\ell(x\infty)=x\Mod{7}.$ Then $B_{22}=B_{\sigma}\cup B_{1}\cup B_{2}\cup B_{3}\cup B_{4}$ has the same properties as $B_{21}$ from Construction \ref{K21design} with $L^{*}$ as defined in this construction.
    Therefore,

    \begin{center}
    the existence of such subgraphs $F_{\sigma},F_{1},F_{2},F_{3},F_{4}$ give $B_{22}=B_{\sigma}\cup B_{1}\cup B_{2}\cup B_{3}\cup B_{4}$, an $F-$decomposition of $K_{22}.$
    \end{center}
\end{construction}
\newpage
\begin{construction}\label{7,8mod14}
    Let $F$ be a forest on seven edges and consider $K_{14t+7}$ and $K_{14t+8}$ for some $t > 1\in \NN$. By Observation \ref{Knlen}, $K_{14t+7}$ has edge-lengths in $L = [1,7t+3]$.
    Let us define the following edge length intervals: $L_{\sigma}=[1,7],\;I_{0}=[8,10]$ and $I_{j}=(I_{\sigma}+10)+7(j-1)=[11+7(j-1),17+7(j-1)]$ for all $0<j\leq t$. Lastly let $L_{\sigma}^{*}=\bigcup_{0<j\leq t} I_{j}$ and $L^{*}=[8,10]$.

    Then if there exists a $\sigma^{+-}-$labeling $F_{\sigma}$ of $F$, we let $B_{\sigma}$ be defined as in Constructions \ref{K21design} and \ref{K22design}, and notice that all edges from $K_{14t+7}$ and $K_{14t+8}$ with lengths in $I_{\sigma}$ are contained in the union of the members of $B_{\sigma}$. Next, recall the vertex partition of $F_{\sigma}$ where all for all $u\in A$ and $v\in B$ with $uv\in E(F_{\sigma})$, $u<v$ and additionally $\ell(uv)=|u-v|$. 

    Let $\varphi_{j}:=v\mapsto v+10(j-1)$ for all $0<j\leq t$ and let $F^{j}_{\sigma}$ simply be $F_{\sigma}$ with $\varphi_{j}$ on $B$ and subsequently each $ab\mapsto a\varphi_{j}(b)$. Also let $B_{\sigma}^{j}=\langle F^{j}_{\sigma}\rangle_{\phi_{7}}$, for each $0<j\leq t$ Recall that since $F_{\sigma}$ contains no wraparound edges in in $K_{21}$, for any edge $ab$ with length $|a-b|\in I_{\sigma}$, $|a-\varphi_{j}(b)|\in I_{j}$. So then since $uv$ is a wraparound edge of $K_{14t+7}$ and $K_{14t+8}$ if and only if $|a-b|>7t+3$, the maximal edge length, $F^{j}_{\sigma}$ must not contain any wraparound edges for each $0<j\leq t$. So then each $F^{j}_{\sigma}$ must contain exactly one edge of each length in $I_{j}$. Therefore for any $1<j\leq t$, since $|B_{\sigma}^{j}|=14t+7$, $B_{\sigma}^{j}$ must contain all edges $K_{14t+7}$ and $K_{14t+8}$ with lengths in $I_{j}$. Therefore $B_{\sigma}^{*}=\bigcup_{0<j\leq t} B_{\sigma}^{j}$ must contain all edges of $K_{14t+7}$ and $K_{14t+8}$ lengths in $L_{\sigma}^{*}$ in the union of it's members. 

    Now, if $\ell\times \ell_{7}-$wise edge-disjoint subgraphs isomorphic to $F$ with the properties defined at end of Constructions \ref{K21design} and \ref{K22design} exist in $K_{14t+7}$ and $K_{14t+8}$, respectively, let $B_{1},B_{2},B_{3}$ be defined as in Construction \ref{K21design} and $\beta_{i}= B_{i}$ from Construction \ref{K22design} for $i=1,2,3,4$. Finally, let $B_{14t+7}=B_{\sigma}\cup B_{\sigma}^{*}\cup B_{1}\cup B_{2}\cup B_{3}$ and let $B_{14t+8}=B_{\sigma}\cup B_{\sigma}^{*}\cup \beta_{1}\cup \beta_{2}\cup \beta_{3}\cup \beta_{4}$. Notice that $B_{14t+7}$ contains all edges in $K_{14t+7}$ of lengths in $L= L_{\sigma}\cup L_{\sigma}^{*}\cup L^{*}$ and $B_{14t+8}$ contains all edges in $K_{14t+8}$ of lengths in $L\cup \{\infty\}=L_{\sigma}\cup L_{\sigma}^{*}\cup L^{*}\cup \{\infty\}$. Therefore since all $G\in B_{14t+7}\cup B_{14t+8}$ are isomorphic to $F$ and edge-disjoint,

    \begin{center}
    the existence of such subgraphs give $B_{14t+7}$ and $B_{14t+8},F-$decompositions of $K_{14t+7}$ and $K_{14t+8}$, respectively.
    \end{center}
    
\end{construction}

\begin{thm} \label{wraplarger}
In $K_{n}$, $uv$ is a wraparound edge if and only if the absolute difference of its endpoints is greater than the maximal length in $K_{n};\floor{\frac{n}{2}}<|u-v|$. 
\begin{proof}
    Let $uv$ be an edge in $K_{n}$ via $\ell$ defined previously. If $uv$ is a wraparound edge, then $n-|u-v|<|u-v|$. So then $\frac{n}{2}-\frac{|u-v|}{2}<\frac{|u-v|}{2}$, and therefore $\floor{\frac{n}{2}}\leq \frac{n}{2}<|u-v|$. If $\floor{\frac{n}{2}}\leq \frac{n}{2}<|u-v|$, note that without loss of generality $u<v$. Then necessarily $n-|u-v|=n-(v-u)<v-u=|u-v|$, so then $n<2(v-u)$ and $\floor{\frac{n}{2}}\leq\frac{n}{2}<(v-u)<|u-v|.$

    Thus,
    $$\floor{\frac{n}{2}}<|u-v|\Longleftrightarrow\text{ uv is a wraparound edge.}$$
\end{proof}
\end{thm}
\newpage
\begin{thm}\label{genmaps}
    Let us refer to $K_{n}$ whose vertices we take to be $\ZZ_{n}$ and $\ZZ_{n}\cup \{\infty\}$ when $n\equiv 7\,\text{ or }\,8\Mod{14}$, respectively, as simply $K_{n}$. Next let us define the edge operation $\ell_{n}:=\ell_{n}(uv)=\mathrm{min}\{|u-v|,n-|u-v|\}$, and let $t>1,\;m>0$ with $h = 14(t-1)$. For wraparound edge $ab$ in $K_{21}$, without loss of generality $a<b$ and\begin{align*}
        &\ell_{14t+7}[(a-mh)\;(b-(m-1)h)]=\ell_{14t+7}[(a+(m-1)h\;(b+mh))]=\ell_{21}[ab],\\
        &(a-mh)+(b-(m-1)h) \equiv (a+(m-1)h + (b+mh)) \equiv a+b\Mod{7}.\\
    \end{align*} Note that $K_{14t+7}$ and $K_{14t+8}$ share vertex labels with the exception of $\infty$ in $K_{14t+8}$ and both come equipped with the $\ell_{14t+7}$ for all non-$\infty$ edges.

    \begin{proof}
        Consider some wraparound edge $ab\in K_{21}$ with $\ell_{21}(ab)=\ell_{ab}$, without loss of generality $a<b$ and $|a-b|=b-a$. Then by definition, $\ell_{ab}=21-(b-a)$ and $\ell_{ab}=(21+a)-b$. Note that $a<b<21\Rightarrow b<21+a$. Suppose $a>13$, then $|a-b|=b-a<21-13=8<10$, the maximal length in $K_{21}$, so by Theorem \ref{wraplarger}, $ab$ is not a wraparound edge, a contradiction. Therefore, $a\leq 13.$

        Let $t>1,\;h = 14(t-1)$ and let $\alpha = a-h,\beta=b\in V(K_{14t+7})$. Well, $(14t+7)-(14(t-1))=14(t-(t-1))+7=21$, so $21+a = (14t+7)-(14(t-1))+a = a-h\in V(K_{14t+7})$. So then since $a\leq 13$, $21+a\leq 34<14t+7$. So, $|\alpha-\beta| = |(21+a)-b|=|\ell_{ab}|=\ell_{ab}.$ Now recall that $\ell_{ab}\leq \floor{\frac{21}{2}}=10,$ the maximal length in $K_{21}$. Therefore, $|\alpha-\beta|\leq 10 = \floor{\frac{14(1)+7}{2}}<\floor{\frac{14t+7}{2}}$ and so by Theorem \ref{wraplarger}, $ \floor{\frac{14t+7}{2}}\not<|\alpha-\beta|\Rightarrow \alpha\beta$ is not a wraparound edge and therefore by definition $\ell_{14t+7}(\alpha\beta) = |\alpha-\beta|=\ell_{ab}= \ell_{14t+7}[(a-h)\;b] = \ell_{21}(ab)$. 
        
        Now let $\alpha' = a, \beta' = b+h\in V(K_{14t+7})$. Well, $\alpha'<\beta'$ and so $|\alpha'-\beta'|=\beta'-\alpha'=b+h-a=(b-a)+h$. Since, $ab$ is a wraparound edge in $K_{21}$, $b-a>10$, the maximal length. So $10+14(t-1)=14t-4<|\alpha'-\beta'|$. $(\floor{\frac{14n+7}{2}})\text{ and }(\floor{\frac{28n-8}{2}})$ are arithmetic sequences with increments $7$ and $14$, respectively, which are equal at $n=1$. So for $t>1$, $\floor{\frac{14t+7}{2}}<\floor{\frac{28t-8}{2}}=14t-4<|\alpha'-\beta'|$ and $\alpha'\beta'$ is a wraparound edge. Therefore, $\ell_{14t+7}(\alpha'\beta')=14t+7-|\alpha'-\beta'|=14t+7-((b-a)+h)=(14t+7)-(14(t-1))-(b-a)=21-(b-a)=\ell_{ab}=\ell_{21}(ab)$ as well. Now by \cite{FreyTran}, for any edge $uv\in E(K_{n}),$ edge length via $\ell_{n}$ is preserved under the operation $uv\mapsto (u+1)\;(v+1)$ modulo $n$. So then the edge lengths of $\alpha\beta$ and $\alpha'\beta'$ are also preserved under $\theta:=uv\mapsto (u-(m-1)h)\;(v-(m-1)h)$ and $\theta':=uv\mapsto (u+(m-1)h)\;(v+(m-1)h)$, respectively.
        
        Finally, $(a-mh)+(b-(m-1)h)=a+b+((m-1)-m)(t-1)14\equiv a+b+((m-1)+m)(t-1)14\equiv (a+(m-1)h + (b+mh))\equiv a+b\Mod{7}$. So then,
        \begin{align*}
            \alpha\beta\overset{\theta}{\mapsto} (\alpha-(m-1)h)\;(\beta-(m-1)h)=(a-mh)\;(b-(m-1)h),\\
            \alpha'\beta'\overset{\theta'}{\mapsto} (\alpha'+(m-1)h)\;(\beta'+(m-1)h)=(a+(m-1)h)\;(b+mh)
        \end{align*}
        preserves edge length and sum modulo 7 and therefore the statement is proven.
        
    \end{proof}
\end{thm}
\newpage
\begin{thm}\label{safemapping}
    Let us refer to $K_{n}$ whose vertices we take to be $\ZZ_{n}$ and $\ZZ_{n}\cup \{\infty\}$ when $n\equiv 7\,\text{ or }\,8\Mod{14}$, respectively, as simply $K_{n}$. Next, let $a,b$ be distinct vertices in $K_{21}$ with $a<b$. 
    
    If $t>1,\;h = 14(t-1)$ and $\alpha = a-h,\;\beta  = b+h\in V(K_{14t+7})$, then

        $$b-a\neq 7\text{ or }b\not\equiv a\Mod{7}\Rightarrow \alpha\neq \beta.$$

    Note that since we let $V(K_{14t+8})=V(K_{14t+7})\cup \{\infty\}$ this statement also holds for all non-$\infty$ vertices in $K_{14t+8}$.

    \begin{proof} Recall that since $a,b\in \ZZ_{21},a<b,$ and they are distinct, $1\leq b-a\leq 20$.
    
    If $b-a\neq 7$, suppose $\alpha=\beta$. Then in $\ZZ_{14t+7}$: $a-14(t-1)=b+14(t-1)$ and so $b-a\equiv -28(t-1)\Mod{14t+7}$. Note that in $\ZZ_{14t+7}$: $-28=14t+7-(28)=14t-21=7(2t-3)\Rightarrow b-a\equiv 7(2t-3)(t-1)\Mod{14t+7}$.
    
    So if $t=2$, then $b-a \equiv 7\Mod{14t+7}$ and so $b-a=7$ or $b-a\geq 35+7=42$, both contradictions. If $t>2$, since $-28\equiv 21\Mod{14(3)+7}$ and $(7(2n-3))_{n>2}$ is strictly increasing, $b-a\geq 21$, a contradiction. So $b-a\neq 7\Rightarrow \alpha\neq \beta$.

    If $b\not\equiv a\Mod{7}$, then suppose $\alpha=\beta$. By the sequence $(7(2n-3))_{n>1},\;b-a\equiv  7(2t-3)(t-1)\Mod{14t+7}$ and so $a-b\equiv 0\Mod{7}$, a contradiction. So $b\not\equiv a\Mod{7}\Rightarrow \alpha\neq \beta$. Therefore the statement is proven.
    \end{proof}
\end{thm}

\begin{corollary}\label{newvertices}
    Let $ab$ be a wraparound edge in $K_{21},\;t>1$ and $h=14(t-1)$ such that $a<b$. Then in $K_{14t+7}:\;a-h\geq 21$ and $b+h>21$. Next, let $u,v\in K_{21}$ with $v\in [u]_{7}$. If $t=2$, then $|u-v|\neq 14\Rightarrow u\pm h\neq v$ and $v\pm h \neq u$. If $t>2$, $u\pm h\neq v$ and $v\pm h\neq u$ in $K_{14t+7}$.
\end{corollary}
\begin{proof}
    In the proof of Theorem \ref{safemapping} it is shown that $a-h=21+a$, so then since $0\leq a\leq 13$, $21\leq 21+a=a-h$ in $K_{14t+7}$. Now, suppose $b\leq 7$. Then since $a<b\leq 7$ and $|a-b|=b-a$, $1\leq |a-b|\leq 7<10$, the maximal length in $K_{21}$. But then by Theorem \ref{wraplarger}, $ab$ is not a wraparound edge, a contradiction. So $b>7$. Therefore $b+h>7+h\geq 21$.

    Next if $t=2$, then $h=14$. So if $u+h = v$ or $u-h=v$, $|u-v|=h$. On the other hand if $|u-v|=h$, then $u-v=h$ or $u-v=-h$ so $u+h = v$ or $u-h=v$. 

    Lastly if $t>2$, then recall that $14t+7=21+14(t-1)=21+h$ so then since $0<u,v<21$, $u+h,v+h<21+h=14t+7$. So then $u+h,v+h\not\in \ZZ_{21}$ and so necessarily $u+ h\neq v$ and $v+ h\neq u$. Now, since $14t+7-h=21$, $14t+7+(v-h)\equiv 21+v\Mod{14t+7}$ and similarly $14t+7+(u-h)\equiv 21+u\Mod{14t+7}$, $u-h,v-h$ are simply $21+u$ and $21+v$ in $K_{14t+7}$, respectively. Lastly, Since $u,v<20$ and $49\leq 14t+7$ we must have that $21\geq 21+u,21+v<20+21=41<49\leq 14t+7$. So then $u-h,v-h\not\in K_{21}$ and therefore necessarily $u-h\neq v$ and $v-h\neq u$. So the statement is proven.

\end{proof}

\begin{center}
    \textbf{Summary of 2.8 through 2.10}
\end{center}

I do everything on an edge to edge basis. Similarly, in the last theorems I am basically creating pairwise vertex criteria to list for our blocks to say that they work.

Theorem \ref{genmaps} tells us that the vertex mappings in our general constructions preserve edge length and sums modulo 7 from $K_{21}$ to larger members of the family. I prove that for any wraparound edge in $K_{21}$, my mapping does this. Therefore we can extend this to our wraparound edges from $K_{22}$ as well.

Theorem \ref{safemapping} tells us that even if we have mapped vertices in the same equivalence class modulo $7$, as long as their absolute difference is not $7$ they will not map to the same vertex. I did not explicitly state that if we add or subtract our fixed increment $h$ uniformly to members in the same equivalence class modulo $7$ this won't happen because this is obvious and even should be given via the $v\mapsto v+1$ being an automorphism.  

Corollary \ref{newvertices} states that these mappings take wraparound edge incident vertices from $K_{21}$ to 'new' vertices; the mapped vertices will never be elements in $\ZZ_{21}$. So for example if we have not touched $0$ but then $14$ is a member of a wraparound edge and we send $14$ somewhere, it can't be anywhere in $\ZZ_{21}$. Additionally it has a weaker statement: if we map a vertex in the same equivalence class modulo $7$ as one we don't map anywhere, they will not map to the same vertex as long as their difference is greater than $14$. This is important for the case where we are forced to send a vertex somewhere and it must also take it's non-wraparound neighbors on the journey as well. This entire corollary is pretty much to deal with the small size of $K_{35}$ relative to our increment. 

The first part of Corollary \ref{newvertices} is to deal with the situation where we have a wraparound vertex mapped somewhere, and then another vertex which is untouched and both are equivalent modulo $7$. The second part ('Next,let...') is to deal with the situation where we have an untouched vertex, and a vertex which is mapped somewhere but which isnt a member of a wraparound edge in $K_{21}$. It's important to distinguish this case, because wraparound edge vertices are necessarily bounded more strictly than just any vertex.

So to summarize the summary, basically if we have representative labelings that generate the blocks we can put together as outlined in Constructions \ref{K21design} and \ref{K22design} to get a decomposition for $K_{21}$ and $K_{22}$, respectively, if we can apply my algorithm (vertex mappings) such that the above conditions hold: length, sum modulo $7$, and isomorphism to the specified graph are preserved from our $K_{21}$ and $K_{22}$ designs and therefore we have the decompositions for the whole family.
\newpage

\begin{center}
\textbf{Generalized Versions}
\end{center}
