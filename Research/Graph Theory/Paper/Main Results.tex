\begin{thm} For any $7-$edge forest $F$, there exists an $F-$decomposition of $K_{21}$ and $K_{22}$ as well as $K_{14t}$ and $K_{14t+1}$ where $t$ is a positive integer.

    \begin{proof}
        We direct the reader to Section A of the appendix. Set $t=1$. For each $7-$edge forest $F$, we present three $\ell\times \ell_{7}$-wise edge- disjoint subgraphs $F_{1}, F_{2}, F_{3}$ with only edges in $\{8,9,10\}$ and a $\sigma^{+-}$ labeling $F_{\sigma}$ at the bottom. So then by Theorem \ref{sigmadesign} there exists an $F-$decomposition of $K_{2(7)r}$ and $K_{2(7)r+1}$ for every positive integer $r$. Additionally, by Construction \ref{K21design}, the existence of these subgraphs gives $B_{21}=B_{1}\cup B_{2}\cup B_{3}\cup B_{\sigma}$, an $F-$decomposition of $K_{21}$ with blocks $B_{i}=\langle F_{i}\rangle_{\phi_{7}}$ for $i=1,2,3$ and $B_{\sigma}=\langle F_{\sigma}\rangle_{\phi_{1}}$. The same holds for $t=1$ in section B of the appendix where we have four $\ell\times \ell_{7}$-wise edge- disjoint subgraphs $F_{1}, F_{2}, F_{3},F_{4}$ with only edges in $\{8,9,10\}\cup \{\infty\}$ and a $\sigma^{+-}$ labeling $F_{\sigma}$ at the bottom. By Construction \ref{K22design}, $B_{22}=B_{1}\cup B_{2}\cup B_{3}\cup B_{4}\cup B_{\sigma}$ is the $F-$decomposition of $K_{22}$ with blocks $B_{i}=\langle F_{i}\rangle_{\phi_{7}}$ for $i=1,2,3$ and $B_{\sigma}=\langle F_{\sigma}\rangle_{\phi_{1}}$. So the statement is proven. 
        
    \end{proof}
    
\end{thm}

\begin{thm}
    For any $7-$edge forest $F$, there exists an $F-$decomposition of $K_{14t+7}$ and $K_{14t+8}$ where $t$ is a positive integer.

    \begin{proof}
        Notice the following properties of any labeling $F$ with only lengths in $\{8,9,10\}\cup \{\infty\}$:
        \begin{align*}
        &(i)\text{ Any wraparound edge for }t=1\text{ is of the form $a(b+h)$ or $(a-h)b$.}\\
        &(ii)\text{ For any vertices of the form }a-h,b+h\text{ where }a\in [b]_{7}\text{ and }a<b: a-b\neq 7.\\
        &(iii)\text{ For any vertices }u,v\pm h,\;|u-v|\neq 14.\\
        &(iv)\text{ Any wraparound edge for }t=1\text{ is of the form }(a-h)b,a(b+h)\text{if and only if }a<b.
        \end{align*}
        By $(i),(iv)$ and Theorem \ref{genmaps}, edge lengths and sums modulo $7$ are preserved by the vertex mappings on our blocks from the $K_{21}$ and $K_{22}$ construction which is simply our blocks where $t=1$. By $(ii)$ and Theorem \ref{safemapping}, any two vertices of forms $a-h,b+h$ are not equal in $K_{14t+7}$ or $K_{14t+8}$. Lastly, by $(iii)$ and Corollary \ref{newvertices}, no two vertices of the form $a,b\pm h$ are equal in $K_{14t+7}$ or $K_{14t+8}$. So then we see that edge length and sums modulo $7$ from our $K_{21}$ and $K_{22}$ constructions are preserved in the blocks for $K_{14t+7}$ and $K_{14t+8}$, respectively. Most importantly, since no two vertices map to the same place in $14t+7$ and $14t+8$ for $t>1$, the isomorphism between $F$ and our blocks is also preserved. Note that no infinity edge is a wraparound edge.

        Therefore, the labelings in the appendix indeed satisfy the properties described in Construction \ref{7,8mod14}. Therefore $B_{14t+7}$ and $B_{14t+8}$ generated by these labelings in section A and B are $F-$decompositions of $K_{14t+7}$ and $K_{14t+8}$, respectively. So the statement is proven.
    \end{proof}
\end{thm}