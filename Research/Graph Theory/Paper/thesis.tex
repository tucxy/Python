%\documentclass[a4paper,10pt]{article}
%\usepackage{a4}
\documentclass[addpoints,11pt]{exam}
\usepackage{amsfonts}
\usepackage{amsthm}
\usepackage{amssymb}
\usepackage{graphicx, enumerate}
\usepackage{amsmath}
\usepackage{latexsym}
\usepackage{longtable}
\usepackage{tabularx}
\usepackage{setspace}
\usepackage{float}
\usepackage{rotating}
\usepackage{tikz}
\usepackage{verbatim}
\usepackage{xcolor}
\usetikzlibrary{calc}
\usepackage{caption}
\usepackage{mathtools}


%\begin{comment}
\newtheorem{theorem}{Theorem}[section]
%\newtheorem{thm}[theorem]{Theorem}%[section]
%\newtheorem{cl}[theorem]{Claim}
\newtheorem{lemma}[theorem]{Lemma}
\newtheorem{prop}[theorem]{Proposition}
%\newtheorem{oprb}[theorem]{Open Problem}
\newtheorem*{oprb}{Open Problem}
\newtheorem{obs}[theorem]{Observation}
%\newtheorem{cor}[theorem]{Corollary}
%\newtheorem{conj}[theorem]{Conjecture}
\newtheorem{const}[theorem]{Construction}
\newtheorem{algo}[theorem]{Algorithm}
%\newtheorem{dfn}[theorem]{Definition}
%\newtheorem{dfn}[theorem]{Definition}
\newtheorem{exm}[theorem]{Example}
%\newtheorem{prm}[theorem]{Problem}
\newtheorem{rem}[theorem]{Remark}
\newtheorem*{rem*}{Remark}
\DeclareMathAlphabet{\mathcal}{OMS}{cmsy}{m}{n}

\theoremstyle{definition}
\newtheorem{definition}[theorem]{Definition}

%\end{comment}


\DeclarePairedDelimiter\ceil{\lceil}{\rceil}
\DeclarePairedDelimiter\floor{\lfloor}{\rfloor}
\newcommand{\rd}{{\rm d}}
\newcommand{\dist}{{\rm{dist}}}


\def\amb{\allowbreak}
\def\ni{\noindent}
\footskip=30pt
\vspace{5cm}
%\begin{document}


%colors
\definecolor{darkpastelgreen}{rgb}{0.01, 0.75, 0.24}
\definecolor{blue-violet}{rgb}{0.54, 0.17, 0.89}
\definecolor{dgreen}{RGB}{20,100,10}

\newcommand{\bblue}[1]{{\color{blue}{#1}}}
\newcommand{\ggreen}[1]{{\color{green}{#1}}}
\newcommand{\dpgreen}[1]{{\color{darkpastelgreen}{#1}}}
\newcommand{\rred}[1]{{\color{red}{#1}}}
\newcommand{\mmag}[1]{{\color{magenta}{#1}}}
\newcommand{\bv}[1]{{\color{blue-violet}{#1}}}


\newcommand{\dal}[1]{\par\vskip2mm\noindent\bblue{Dalibor: #1}\par\noindent}
\newcommand{\syl}[1]{\par\vskip2mm\noindentGreen{Sylwia: #1}\par\noindent}
\marginparwidth1.5in
%need to do \margbl{}{my own text} at least in some installations --- leave blank {} right after command



\newcommand{\dalb}[1]{\par\vskip2mm\noindent\bblue{Dalibor: #1}\par\noindent}
\newcommand{\dalr}[1]{\par\vskip2mm\noindent\rred{Dalibor: #1}\par\noindent}
\newcommand{\dalm}[1]{\par\vskip2mm\noindent\mmag{Dalibor: #1}\par\noindent}
\newcommand{\dalgr}[1]{\par\vskip2mm\noindent\dpgreen{Dalibor: #1}\par\noindent}
\newcommand{\dalbv}[1]{\par\vskip2mm\noindent\bv{Dalibor: #1}\par\noindent}


%http://joshua.smcvt.edu/latex2e/_005cnewcommand-_0026-_005crenewcommand.html

\marginparwidth1.3in
\newcommand{\margbl}[1]{\marginpar{\textcolor{blue}{#1}}}
\newcommand{\margred}[1]{\marginpar{\textcolor{red}{#1}}}
\newcommand{\marggr}[1]{\marginpar{\textcolor{green}{#1}}}

%\usepackage{parskip}

\newcommand{\GG}{\ensuremath{\mathcal{G}}}
\newcommand{\BB}{\ensuremath{\mathcal{B}}}
\newcommand{\ZZ}{\ensuremath{\mathbb{Z}}}
\newcommand{\NN}{\ensuremath{\mathbb{N}}}
\parindent0pt
\parskip4pt

%%% Formatting: Page Header
\newcommand{\StudentName}{Danny Banegas}
\newcommand{\AssignmentName}{Materials}
\newcommand{\CourseName}{Thesis}

\pagestyle{headandfoot}
\runningheadrule
\firstpageheadrule
\firstpageheader{\CourseName}{\StudentName}{\AssignmentName}
\runningheader{\CourseName}{\StudentName}{\AssignmentName}
\firstpagefooter{}{\thepage}{}
\runningfooter{}{\thepage}{}

\printanswers

\begin{document}

\begin{definition}\label{sigmapm}
  \cite{sigma} A $\sigma^{+-}$ labeling of $G$ is an injection $f:V(G) \rightarrow \{0, 1,\hdots, 2m-2\}$ which induces a bijective length function $\ell:E(G)\rightarrow \{1, 2,\hdots, m\}$ where $\ell(uv) = f(v)-f(u)$ for every edge $uv\in E(G)$ with $u\in A$ and $v\in B$, and $f$ has the additional property that $f(u)-f(v) \neq m$ for all $u\in A$ and $v\in B$.
\end{definition}

\begin{theorem}\label{sigmadec}
  (Freyberg and Tran, \cite{sigma}). Let $G$ be a bipartite graph with $m$ edges and a $\sigma^{+-}$-labeling such that the edge of length $m$ is a pendant edge $e$. Then there exists a $G$-decomposition of $K_{2mr}$ and $K_{2mr+1}$ for every positive integer $r$.
\end{theorem}

\begin{const}\label{8910Bmod21}
Let $F$ be a forest graph on seven edges, and consider $K_{21}$. Recall that any $K_{n}$ via $\ell$ has edge lengths in $L = \{1,2,\hdots, \floor{\frac{n}{2}}\}$. If $B_{\sigma}$ is the $\sigma^{+-}$ labeling of $F$, then by Theorem \ref{sigmadec} every edge of length in $L_{\sigma}=\{1,2,\hdots, 7\}\subset L$ is generated by $B_{\sigma}$ via $\phi_{1}:=v\mapsto v+1$ on $V(B_{\sigma})$. Let $\BB_{\sigma}=\langle B_{\sigma}\rangle_{\phi_{1}}$. So then $F\setminus \bigcup\limits_{G\in \BB_{\sigma}} G$ contains all edges with lengths in $L_{\sigma}$.

Recall that in any $K_{n}$ there exist $n$ edges of each distinct length, but every edge is also uniquely determined by it's endpoints. If we partition the vertices of $K_{21}$ modulo 7, naturally this induces a singleton edge partition modulo $7\times 7$. We can consolidate this partition via $\ell_{7}:=\;ab\mapsto a+b\;(mod\;7)$ on the edges. So now we see that in partitioning the edges of $K_{21}$ via $\ell_{7}$ and $\ell$, each partite set $P_{i,j}$ will have three edges of the same length which also belong to the same equivalence class with respect to $\ell_{7}$ where $(i,j)\in L\times \ZZ_{7}$. The result is that $L_{i}=\bigcup\limits_{j\in \ZZ_{7}}P_{i,j}=\{uv\in E(K_{21}) \mid \ell(uv)=i \}$.

Let $\phi_{7}:=v\mapsto v+7$, $B_{i,j} \subset K_{21}$ be a subgraph which contains some edge $ab$ with $(\ell\times\ell_{7}\;(ab))=(i,j)$, and $\BB_{i,j}=\langle B_{i,j}\rangle_{\phi_{7}}$. Then if $\mathcal{H}_{i,j}=\bigcup\limits_{G\in \BB_{i,j}} G$ we must have that $P_{i,j}\subset E(\mathcal{H}_{i,j})$. Therefore, we must have that $L_{i}\subset E(\bigcup\limits_{j\in \ZZ_{7}}\mathcal{H}_{i,j})=\{uv\in E(K_{21})\mid \ell(uv)=i\}$.

So if there exist edge-disjoint blocks $B_{1},B_{2},B_{3} \cong F$ with only edges of lengths in $L^{*}$, let $\BB_{i}=\langle B_{i}\rangle_{\phi_{7}}$ for $i\in \{1,2,3\}$. So $\bigcup\limits_{i\in\{1,2,3\}}\;\BB_{i}$ must contain all edges of lengths in $L^{*}$ in $K_{21}$ and therefore,

$$\text{the existence of such blocks give }\BB_{21}=\BB_{\sigma}\cup\BB_{1}\cup\BB_{2}\cup\BB_{3} \text{, an F-design of order 21}.$$

\end{const}

\begin{const}\label{8910Bmod22}
Let $F$ be a forest graph on seven edges and consider $K_{22}$. We let $B_{\sigma}$ refer to the same block as previously described in the construction for $F-$designs of order $21$. Similarly to in the previous construction, if $B_{1},B_{2},B_{3},B_{4}$ are edge-wise disjoint blocks which are isomorphic to $F$ with only edges of lengths $\{8,9,10\}\cup \{\infty\}$ via $\phi_{7}$ where we take $\infty = 0$ with respect to computing $\ell_{7}$. Then using the previous conventions for $\BB_{i}$ where $i\in \{1,2,3,4,\sigma\}$,

$$\text{the existence of such blocks give }\BB_{22}=\BB_{\sigma}\cup\BB_{1}\cup\BB_{2}\cup\BB_{3}\cup\BB_{4} \text{, an F-design of order 22}.$$
\end{const}
\newpage

\begin{definition}
  Consider any edge $uv\in E(K_{n})$. We say:

  \begin{align}
  uv\text{ is a wraparound edge if }\ell(uv)=n-|u-v|\\
  uv\text{ is a short edge if }\ell(uv)=|u-v|\\
  \end{align}.

  Now for any edge $uv\in E(K_{n})$, without loss of generality $v<u$ and we say:   

  \begin{align}
    v\text{ is a wraparound vertex if }uv\text{ is a wraparound edge}\\
    u\text{ is a short vertex if }\forall ux\in E(K_{n}), ux\text{ is a short edge}
  \end{align}.

  
\end{definition}

\begin{algo}
Let $F$ be a forest on seven edges, $\BB_{21},\BB_{22}$ be the $F-$decompositions of $K_{21},K_{22}$, respectively, given by Constructions \ref{8910Bmod21} and \ref{8910Bmod22}, and consider $K_{14t+7}$ along with $K_{14+8}$ where we take the vertices to be members of $\ZZ_{21}\cup \{\infty\}$ in the natural way (according to $C_{14t+7}$ and $C_{14+8}$ where $22 \mapsto \infty$ along with all it's incident edge lengths).

For each block in 
\end{algo}

\begin{const}
Let $F$ be a forest on seven edges and consider $K_{14t+7}$ and $K_{14t+8}$ for $t > 1\in \NN$. By definition, $K_{14t+7}$ has edge-lengths $[1,7t+3]$.
Let us define the following edge length intervals: $I_{\sigma}=[1,7],\;I_{0}=[8,10]$ and $I_{j}=(I_{\sigma}+10)+7(j-1)=[11+7(j-1),17+7(j-1)]$ for all $j>0$.

So $L=\bigcup\limits_{0\leq j\leq t}\;I_{j}\cup I_{\sigma}$ is the set of all distinct lengths in $K_{14t+7}$ and $L\cup \{\infty\}$ is the set of all distinct lengths in $K_{14t+8}$ for $t>1$. 
\end{const}

\begin{thebibliography}{99}

  \bibitem{sigma}
  B. Freyberg and N. Tran, Decomposition of complete graphs into bipartite unicyclic graphs with eight edges, J. Combin. Math. Combin. Comput., \emph{114} (2020), 133-142.

  \bibitem{tripartite}
R. C. Bunge, A. Chantasartrassmee, S.I. El-Zanati, and C. Vanden Eyn-den, On cyclic decompositions of complete graphs into tripartite graphs, \emph{J. Graph Theory} \textbf{72} (2013), 90--111.

  \end{thebibliography}

\end{document}