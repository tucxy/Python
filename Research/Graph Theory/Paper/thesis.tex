%\documentclass[a4paper,10pt]{article}
%\usepackage{a4}
\documentclass[addpoints,11pt]{exam}
\usepackage{amsfonts}
\usepackage{amsthm}
\usepackage{amssymb}
\usepackage{graphicx, enumerate}
\usepackage{amsmath}
\usepackage{latexsym}
\usepackage{longtable}
\usepackage{tabularx}
\usepackage{setspace}
\usepackage{float}
\usepackage{rotating}
\usepackage{tikz}
\usepackage{verbatim}
\usepackage{xcolor}
\usetikzlibrary{calc}
\usepackage{caption}
\usepackage{mathtools}


%\begin{comment}
\newtheorem{theorem}{Theorem}[section]
%\newtheorem{thm}[theorem]{Theorem}%[section]
%\newtheorem{cl}[theorem]{Claim}
\newtheorem{lemma}[theorem]{Lemma}
\newtheorem{prop}[theorem]{Proposition}
%\newtheorem{oprb}[theorem]{Open Problem}
\newtheorem*{oprb}{Open Problem}
\newtheorem{obs}[theorem]{Observation}
%\newtheorem{cor}[theorem]{Corollary}
%\newtheorem{conj}[theorem]{Conjecture}
\newtheorem{const}[theorem]{Construction}
\newtheorem{algo}[theorem]{Algorithm}
%\newtheorem{dfn}[theorem]{Definition}
%\newtheorem{dfn}[theorem]{Definition}
\newtheorem{exm}[theorem]{Example}
%\newtheorem{prm}[theorem]{Problem}
\newtheorem{rem}[theorem]{Remark}
\newtheorem*{rem*}{Remark}
\DeclareMathAlphabet{\mathcal}{OMS}{cmsy}{m}{n}

\theoremstyle{definition}
\newtheorem{definition}[theorem]{Definition}

%\end{comment}


\DeclarePairedDelimiter\ceil{\lceil}{\rceil}
\DeclarePairedDelimiter\floor{\lfloor}{\rfloor}
\newcommand{\rd}{{\rm d}}
\newcommand{\dist}{{\rm{dist}}}
\newcommand{\Mod}[1]{\ (\mathrm{mod}\ #1)}


\def\amb{\allowbreak}
\def\ni{\noindent}
\footskip=30pt
\vspace{5cm}
%\begin{document}


%colors
\definecolor{darkpastelgreen}{rgb}{0.01, 0.75, 0.24}
\definecolor{blue-violet}{rgb}{0.54, 0.17, 0.89}
\definecolor{dgreen}{RGB}{20,100,10}

\newcommand{\bblue}[1]{{\color{blue}{#1}}}
\newcommand{\ggreen}[1]{{\color{green}{#1}}}
\newcommand{\dpgreen}[1]{{\color{darkpastelgreen}{#1}}}
\newcommand{\rred}[1]{{\color{red}{#1}}}
\newcommand{\mmag}[1]{{\color{magenta}{#1}}}
\newcommand{\bv}[1]{{\color{blue-violet}{#1}}}


\newcommand{\dal}[1]{\par\vskip2mm\noindent\bblue{Dalibor: #1}\par\noindent}
\newcommand{\syl}[1]{\par\vskip2mm\noindentGreen{Sylwia: #1}\par\noindent}
\marginparwidth1.5in
%need to do \margbl{}{my own text} at least in some installations --- leave blank {} right after command



\newcommand{\dalb}[1]{\par\vskip2mm\noindent\bblue{Dalibor: #1}\par\noindent}
\newcommand{\dalr}[1]{\par\vskip2mm\noindent\rred{Dalibor: #1}\par\noindent}
\newcommand{\dalm}[1]{\par\vskip2mm\noindent\mmag{Dalibor: #1}\par\noindent}
\newcommand{\dalgr}[1]{\par\vskip2mm\noindent\dpgreen{Dalibor: #1}\par\noindent}
\newcommand{\dalbv}[1]{\par\vskip2mm\noindent\bv{Dalibor: #1}\par\noindent}


%http://joshua.smcvt.edu/latex2e/_005cnewcommand-_0026-_005crenewcommand.html

\marginparwidth1.3in
\newcommand{\margbl}[1]{\marginpar{\textcolor{blue}{#1}}}
\newcommand{\margred}[1]{\marginpar{\textcolor{red}{#1}}}
\newcommand{\marggr}[1]{\marginpar{\textcolor{green}{#1}}}

%\usepackage{parskip}

\newcommand{\GG}{\ensuremath{\mathcal{G}}}
\newcommand{\BB}{\ensuremath{\mathcal{B}}}
\newcommand{\ZZ}{\ensuremath{\mathbb{Z}}}
\newcommand{\NN}{\ensuremath{\mathbb{N}}}
\parindent0pt
\parskip4pt

%%% Formatting: Page Header
\newcommand{\StudentName}{Danny Banegas}
\newcommand{\AssignmentName}{Materials}
\newcommand{\CourseName}{Thesis}

\pagestyle{headandfoot}
\runningheadrule
\firstpageheadrule
\firstpageheader{\CourseName}{\StudentName}{\AssignmentName}
\runningheader{\CourseName}{\StudentName}{\AssignmentName}
\firstpagefooter{}{\thepage}{}
\runningfooter{}{\thepage}{}

\printanswers

\begin{document}

\begin{definition}\label{sigmapm}
  \cite{sigma} A $\sigma^{+-}$ labeling of $G$ is an injection $f:V(G) \rightarrow \{0, 1,\hdots, 2m-2\}$ which induces a bijective length function $\ell:E(G)\rightarrow \{1, 2,\hdots, m\}$ where $\ell(uv) = f(v)-f(u)$ for every edge $uv\in E(G)$ with $u\in A$ and $v\in B$, and $f$ has the additional property that $f(u)-f(v) \neq m$ for all $u\in A$ and $v\in B$.
\end{definition}

\begin{theorem}\label{sigmadec}
  (Freyberg and Tran, \cite{sigma}). Let $G$ be a bipartite graph with $m$ edges and a $\sigma^{+-}$-labeling such that the edge of length $m$ is a pendant edge $e$. Then there exists a $G$-decomposition of $K_{2mr}$ and $K_{2mr+1}$ for every positive integer $r$.
\end{theorem}

\begin{const}\label{K21design}
  Let $F$ be a forest on $7$ edges and consider $K_{21}$. By \ref{Knlen}, any edge in $K_{21}$ has a length in $L=\{1,\hdots,10\}$ via $\ell$ as stated in  Definition \ref{sigmapm}. Next, by Theorem \ref{sigmadesign}, if $F_{\sigma}$ is a $\sigma^{+-}-$labeling of $F$, then $B_{\sigma}=\langle F_{\sigma}\rangle$ via $\phi_{1}:=v\mapsto v+1$ contains all edges of $K_{21}$ with lengths in $L_{\sigma}=\{1,\hdots,7\}$ by the union of it's members. Additionally, all $G\in B_{\sigma}$ are isomorphic to $F$.

  Now, clearly each edge in $K_{21}$ is uniquely determined by it's endpoints. So naturally observe that if we partition the vertices of $K_{21}$ modulo $7$, a singleton edge partition of $K_{21}$ modulo $7\times 7$ is induced. Let $\ell:=ab\mapsto a+b\Mod{7}$. We now partition the edges of $K_{21}$ modulo $\sim$ where $ab\sim cd\text{ if } \ell(ab)=\ell(cd)\text{ and }\ell_{7}(ab)=\ell_{7}(cd)$, and specify each partite set $P_{i,j}$ via $ab\in P_{i,j}$ if $\ell\times \ell_{7}(ab)=(i,j)$. Each $P_{i,j}$ contains three edges in the same equivalence class modulo $\sim$; three edges of the same length $i$ which are in the same endpoint equivalence class modulo $7\times 7$. Lastly, let $L_{i}=\{uv\in E(K_{21})\mid \ell(uv)=i\},\forall i\in L$, then  $L_{i}=\bigcup_{j\in \mathbb{Z}_{7}}P_{i,j}$.

  Let $\phi_{7}:=v\mapsto v+7$ and $H_{i,j}\subseteq K_{21}$ be a subgraph containing some edge $ab$ such that $\ell\times \ell_{7}(ab)=(i,j)$. Then if $G_{i,j}=\bigcup_{G\in \langle H_{i,j}\rangle_{\phi_{7}}}G,\;P_{i,j}\subseteq E(G_{i,j});$ each $G_{i,j}$ contains the partition $P_{i,j}$. Then since for any $(i,j)\in L\times \ZZ_{7},$ such a subgraph $H_{i,j}$ exists, so does such a $G_{i,j}$. Therefore, for any $i\in L$, $L_{i}\subseteq E(\bigcup_{j\in \ZZ_{7}}G_{i,j})$ for such $G_{i,j}$. In other words, we can generate all edges of length $i\in L$ in $K_{21}$ by finding subgraphs containing edges of length $i$ from each distinct equivalence class modulo $7\times 7$ and applying $\phi_{7}$ to their vertices.

  So finally, if there exist edge-disjoint subgraphs $F_{1},F_{2},F_{3}\cong F$ of $K_{21}$ with only edges of lengths in $L^{*}=\{8,9,10\}$, let $B_{i}=\langle F_{i}\rangle_{\phi_{7}}$ for each $i=1,2,3$. Then $\bigcup_{i=1,2,3}B_{i}$ must contain all edges of $K_{21}$ with lengths in $L^{*}$ by the union of it's members. Let $B_{21}=B_{\sigma}\cup B_{1}\cup B_{2}\cup B_{3}$ and recall that $B_{\sigma}$ contains all edges of lengths in $L_{\sigma}$ by the union of it's members. So then since $L=L_{\sigma}\cup L^{*},\;K_{21}=\bigcup_{G\in B_{21}}G,$ and all $G\in B_{21}$ are isomorphic to $F$ and edge disjoint by definition. Therefore,
  \begin{center}
  the existence of such subgraphs $F_{\sigma},F_{1},F_{2},F_{3}$ give $B_{21}=B_{\sigma}\cup B_{1}\cup B_{2}\cup B_{3}$ an $F-$decomposition of $K_{21}.$
  \end{center}

\end{const}

\begin{const}\label{K22design}
  Let $F$ be a forest graph on seven edges and consider $K_{22}$. For this construction we simply take the $22$nd vertex to be $\infty$; we add $\infty$ to the neighborhood of each vertex in $K_{21}$, and let $\ell(x\infty)=\infty$ for all $x\in V(K_{22})$. Next, we let $B_{\sigma}$ refer to the same block as previously described in Construction \ref{K21design}. Similarly to in the previous construction, if $F_{1},F_{2},F_{3},F_{4}\cong F$ are edge-wise disjoint subgraphs of $K_{21}$ with only edges of lengths in $L^{*}=\{8,9,10\}\cup \{\infty\}$, we let $B_{i}=\langle F_{i}\rangle_{\phi_{7}}$ via $\phi_{7}$ as defined in Construction \ref{K21design} with the new condition $\ell(x\infty)=x\Mod{7}.$ Then $B_{22}=B_{\sigma}\cup B_{1}\cup B_{2}\cup B_{3}\cup B_{4}$ has the same properties as $B_{21}$ from Construction \ref{K21design} with $L^{*}$ as defined in this construction.
  Therefore,

  \begin{center}
  the existence of such subgraphs $F_{\sigma},F_{1},F_{2},F_{3},F_{4}$ give $B_{22}=B_{\sigma}\cup B_{1}\cup B_{2}\cup B_{3}\cup B_{4}$, an $F-$decomposition of $K_{22}.$
  \end{center}
\end{const}
\newpage

\begin{definition}
  Consider any edge $uv\in E(K_{n})$. We say:

  \begin{align}
  uv\text{ is a wraparound edge if }\ell(uv)=n-|u-v|\\
  uv\text{ is a short edge if }\ell(uv)=|u-v|\\
  \end{align}.

  Now for any edge $uv\in E(K_{n})$, without loss of generality $v<u$ and we say:   

  \begin{align}
    v\text{ is a wraparound vertex if }uv\text{ is a wraparound edge}\\
    u\text{ is a short vertex if }\forall ux\in E(K_{n}), ux\text{ is a short edge}
  \end{align}.

  
\end{definition}

\begin{algo}
Let $F$ be a forest on seven edges, $\BB_{21},\BB_{22}$ be the $F-$decompositions of $K_{21},K_{22}$, respectively, given by Constructions \ref{8910Bmod21} and \ref{8910Bmod22}, and consider $K_{14t+7}$ along with $K_{14+8}$ where we take the vertices to be members of $\ZZ_{21}\cup \{\infty\}$ in the natural way (according to $C_{14t+7}$ and $C_{14+8}$ where $22 \mapsto \infty$ along with all it's incident edge lengths).

For each block in 
\end{algo}

\begin{const}
Let $F$ be a forest on seven edges and consider $K_{14t+7}$ and $K_{14t+8}$ for $t > 1\in \NN$. By definition, $K_{14t+7}$ has edge-lengths $[1,7t+3]$.
Let us define the following edge length intervals: $I_{\sigma}=[1,7],\;I_{0}=[8,10]$ and $I_{j}=(I_{\sigma}+10)+7(j-1)=[11+7(j-1),17+7(j-1)]$ for all $j>0$.

So $L=\bigcup\limits_{0\leq j\leq t}\;I_{j}\cup I_{\sigma}$ is the set of all distinct lengths in $K_{14t+7}$ and $L\cup \{\infty\}$ is the set of all distinct lengths in $K_{14t+8}$ for $t>1$. 
\end{const}

\begin{thebibliography}{99}

  \bibitem{sigma}
  B. Freyberg and N. Tran, Decomposition of complete graphs into bipartite unicyclic graphs with eight edges, J. Combin. Math. Combin. Comput., \emph{114} (2020), 133-142.

  \bibitem{tripartite}
R. C. Bunge, A. Chantasartrassmee, S.I. El-Zanati, and C. Vanden Eyn-den, On cyclic decompositions of complete graphs into tripartite graphs, \emph{J. Graph Theory} \textbf{72} (2013), 90--111.

  \end{thebibliography}

\end{document}